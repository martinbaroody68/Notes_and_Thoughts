24/09/22

Contradiction in the moral system?

Consider abortion

Aborting is essentially morally equated to shooting a newborn point blank range. Both are the same child, feel pain, have contributed equally to society. Both are killed without any defence on their part.

Wrong because of the “killing” aspect. Why is this wrong from a moral perspective? For a moral to have weight generally it should have some kind of utilitarian aspect for society as a whole or at least an individual, otherwise the moral virtue has no weight unless dictated by a God separate from the universe. Let’s go with the utilitarian perspective. Generally a societally impactful moral is more beneficial for humanity than a purely individually impactful one (notice how selfishness is never considered a virtue).

Then going with this logic, the significance of a moral virtue is correlated with its positive impact on society and humanity.

Once again other sources of significance are an ordering provided by God that have no rational basis according to our universe. Another is that the significance of morals is determined by the individual and there is no universal or generalizable weight on morals at all. This latter perspective is useless, nihilistic, and very detrimental to society and humanity if influential. On top of that it is contrary to evidence. Within societies, individuals tend to order morals similarly, which could show that human moral ordering is mostly determined by society and not individuals. Perhaps this itself is arbitrary but not as arbitrary as the uniformly variable moral ordering hypothesis. Since there exists some societal standards for moral virtues and orderings, we can assume that these morals and ordering would have some non random criteria for being selected, whether rational or not. If over time a society maintains a standard that allows generalization and extrapolation and intuitive decision making according to that standard, it makes sense to assume there is a criteria behind the standard, in other words it doesn’t come out of thin air and there can be a rational decision making process to align with that standard. So we can say with high probability that the societal standard for morals and moral ordering has some basis behind it. In other words the hypothesis that moral ordering is simply random and arbitrary is not sufficient, you could say that it’s in the eye of the beholder but most of the beholders have the same eye.

The point is that within a society we can logically make conclusions about moral orderings with respect to the basis behind the value of a moral. If two morals have the same value we can say that they are equally significant. Or if two actions have the same moral value behind them, the two actions are morally equivalent, equally bad or good. This is all according to the moral standard of the society.

So let’s go back to abortion. We considered abortion to be morally wrong because we logically equated it with shooting a newborn, which is intuitively wrong. The equivalence was held along the basis that both involved killing under the sand circumstances. We could say that the moral value of killing under these circumstances is negative, and it is this that gives the negative moral value to abortion.

What makes this killing morally wrong? To see this we need to investigate the basis behind moral ordering in general. It might be a little fine grained but we can assume as before that there is a correlation with impact on society/humanity/other individuals. What are the ramifications of killing with respect to this? We could say that killing deprives the killed individuals from contributing to society. There are other ramifications as well (creates enemies, causing conflict, etc.) but in the case of abortion the main logical ramification that applies is that the killed fetus has been deprived of a chance to live and contribute to society. This is the logical basis behind the negative moral value of abortion. Saying “abortion is wrong because it involves killing” is not rational in itself because it does not involve the rationale behind what makes the killing wrong with respect to the attribution of values to morals. We need to show that this killing has negative moral value, in other words we need to show that the killing has negative impact on society/humanity/others. Clearly we could simply say that killing the fetus obviously has negative impact on the fetus, but valuation of morals based only on the ramification of the person affected by the action is absurd and ineffective. Defending yourself against a serial killer would hurt the serial killer, so the valuation of your action would be negative according to that standard, but clearly no one believes this, and it in fact leads to a contradiction because the serial killer is hurting you as well. Clearly impact on the affected person in itself is not enough to conclude that an action is morally right or wrong. Impact on society/humanity of the moral was reversed is a far more salient basis for value. Then, from that perspective, the most meaningful negative impact of abortion on a fetus who hasn’t had any chance to do anything meaningful is that the fetus will NEVER have a chance to do anything meaningful that benefits society or humanity. One person can indeed have an important positive impact (or negative impact, but this is more rare assuming the society is working correctly and effectively indoctrinating people into its latent value system).

Aborting the fetus will prevent the positive impact of the fetus. You could also say though that aborting the fetus could have a positive impact on society though, perhaps the parents aren’t ready for it, and by keeping the fetus you have stunted the impact of the parents. Then which is worse? Killing the child or letting it live? Perhaps here, fine grained probability and cumulative value calculations come into play, intuitively the parents can still provide value to society (possibly more than before, considering that they will be raising another kid who can contribute positively, while themselves contributing as well, so the sum can be positive). Cumulatively and in expectation the value is positive.

However reducing the moral value of an action to these impact calculations feels bizarre. By this logic killing a few people for a little end benefit to society is morally encouraged.

Also by this logic the moral value of an action is reduced to the utility it will bring in expectation. What if the actor makes a miscalculation and ends up having a negative impact of society despite his intention to improve society? We descend then into utilitarianism. Even if the actor rationally believed the expected value would be positive, but ended up miscalculating, in other words the true expected value of the action is negative, how do we judge his action? Does intention factor into moral valuation?

Let’s say that it does. Then that’s like saying we give a pass to actions such as killing, deception, ruining of lives, just because the actor believed that it would bring about extra benefit to humanity, even if it didn’t, and in wouldn’t in expectation.

This is obviously absurd. Clearly something has been missed. Perhaps the valuation of an action is some combination of the impact on humanity and the class of action in itself. By class of action, I mean we could divide actions into categories, such as charity, deception, murder, etc. So even if a murder or multiple murders brought about a slight benefit to the state of humanity, the action could still be immoral, because the category of murder has a high negative value associated with it. The category modulates the value of the action. But now it becomes far more complicated to rationally calculate or determine the moral value of an action. Now personal opinions and irrationality start to seep in. No one knows the true value associated with a category, after all. It’s hard enough to approximate the societal impact of an action, now we have significantly increased the complexity. The societal moral ordering becomes hard/impossible to justify. Humans cannot confidently put a moral value on anything.

We could go with the category alone, meaning that killing is always negative, theft is always negative, deception is always negative, etc. Some kind of rule based system. But if we end with categories we have no rational basis behind the moral system. If God created the moral system, this poses no issue but has drastically different implications. If the moral system is created by human societies, it poses problems. Perhaps it could best be described as a naïve system that isn’t always rational and doesn’t always provide the most value for humanity, but in expectation it does provide value for humanity and allows society to function and thrive without chaos. Moral judgements are then based on following these rules. But when two rules come into conflict or a situation exists that doesn’t fit neatly into these inherently limited rules, it is variable human interpretation which is generally irrational that allows people to decide how to act. Since the rules have no deeper value structure associated with them, there is no way to distinguish between which actions are more right or more wrong, or on some cases, whether an action is right or wrong at all. With pure rules always comes ambiguity.

Back to abortion, considering a moral system which involves a basis for moral value based on societal impact, we have already discussed that abortion is wrong because in expectation it is negative for society because it prevents the opportunity of another human to contribute to society. But thinking along that logic, that would mean that not having sex with every woman you connect with is immoral. Think about it. If you were to produce offspring with every woman who liked you, these offspring could all contribute to society, and humans are better in numbers, the more humans the better. In fact, the more times you produce offspring, the better, meaning that producing offspring multiple times with the same woman is better than having only one child or none at all. In other words, not doing this has a lower moral value. By our previous logic, not acting this way is immoral, and Genghis Khan is the most saintly of all. Notice that the impact of not having a lot of sex is that same as abortion. In either case, you stunted the opportunity of a new human to contribute to society. But this is absurd. We could counter this by saying that absence of the best outcome, or acting in a way that won’t maximize the outcome, is not immoral in itself. An action is only immoral if it will end up having a negative implication. We could reframe and say that closing the door on a great opportunity is not immoral as long as we end up doing something good (but not as good) for society in the end. The net value becomes positive, purging the best path does not mean that the end state is not positive. In other words, actions are morally justified if they lead to a positive gain. The previous ideology was more that actions are justified if they are the best action according to the moral ordering. So according to the positive gain basis for moral value, it doesn’t matter if your action is the best action. But by that logic, killing is morally justified. If a killer murders a homeless man who had no significant impact on anyone at the time, the outcome in terms of the state is still positive, because the killer can still contribute to society whereas the death of the homeless man had no real negative impact. In other words the temporal aspect of potential impact is completely ignored, leading to a totally different set of morally justified actions, a lot of which also seek absurd. Going back to abortion, under this value system, abortion is justified because the child has not contributed to anything yet anyway. The net outcome can only be positive. We discard the potential for the child to contribute and only consider the immediate realized impact instead of possible impacts and potential paths.

In other words, human created moral systems end up being absurd, contradictory, and unintuitive, no matter which way you look at them.