..Theft shall not be tolerated

Martin pensees 21/05/23

Power of God - disciple of Lord - it’s own immense power

Everyone has good admirable traits to learn from and integrate

Understand “dependent” entities/complexities by understanding overarching forces that sustain them (for example physics -> chem-> bio -> …). Temporal dependencies work here too.

Habits as latency reduction
        ⁃       and free computation for higher order learning

Parsa’s cool idea — billboard demog targeting with CV

Arrogance and daydreams are linked. They drain motivation. Drain time and energy. Plant and reinforce delusions. In the end the arrogance ironically rends me powerless. The powerlessness and false narratives makes the daydreams more enticing. They are a liar and a thief. A manipulator and infinite plunderer.
        ⁃       but when daydreams/fantasies arise (not ideal btw) they reveal a fusion of a true/positive value and “false values” that deep from the unconscious, some of which keep the narrative stuck in the “fantasy realm” never to be acted upon. In that sense the fantasy intrinsically implicitly  provides clues and pointers of what to deal with and how. Like an error and gradient signal bundled into one entity.

Perception of our own lives as a narrative. How accurate are these narratives? Narrative affects everything else (lens for all other perceptions and fuel for all other actions - like RL with a policy affecting the experiences but also changing itself online based on itself and it’s own experience of actions and states and outcomes/rewards — the “reward function” is itself influenced by the policy/brain — a loop that can easily lead to chaos and bad loops but also loops of positive nature)

Is there a gap/discrepancy between this narrative (semi conscious) and daydreams (mostly unconscious) and a person’s strong view of an “ideal” (mostly conscious, but obviously not entirely based on “rational deduction”) which they truly want to act upon?

Do not put the Lord to the test. There is some process transcending rationality which indicates “true value” despite lack of immediate rational explanation or ordering. Choice: comfort (ignoring “transcendent” intuition, convincing yourself “it’ll be fine, logically…” — btw in itself seeds a mindset prone for eventual failure due to conduciveness for slippery slopes and priority to pleasure) vs. listening and acting upon it, which may be uncomfortable but why not. These illuminations of “true value” don’t hurt in the grand scheme in the end even if “false positive”. Habitual brain comes into play against “transcendental intuition”. Need to foster split second decision making aligned with transcendence.
        ⁃       can align oneself to emphatic action, or begrudging action (view of “this is uncomfortable”). Second view is not sustainable long term. Or even short term most of the time.
        ⁃       Lack of faith and doubt are negative frames of mind which undo what is right in the end. So is a frame of “this is gonna be hard, and has a possibility of failure” ended at that point without strong faith in certainly overcoming the challenge
        ⁃       Do not view direct challenges/tasks/goals in themselves as very hard or very unpleasant or unachievable. Think a level above and beyond, in terms of “effort”. View: enough effort, and true faith, solve these things, with certainty. Effort needed, but a finite amount (however much is necessary). The question should be, “is effort unachievable? Is pain intolerable?” Now we are in a different realm. Of pure willpower rather than correctness or unsatisfiability in a real world. Doubt and difficulty are not really concepts here. The aforementioned question has an easy and intuitive resolution: “no”. It is the pure domain of “free will”. Instantly making the right choice. Of desiring the end more than the alternative. Consistency and success amount to always (or at least “consistently enough”, but NEVER rely on that for aforementioned reasons of slippery slope) making the right choice, looking to and following the right thing rather than its “comfortable” opposite. This is always doable, at all times. Frame of mind should be “I CAN and WILL do the right thing, ALWAYS”. (No grey area for failure to seep through, I.e. no “most of the time”. The idea is strong for a reason. Consciously introduce areas for failure to seep through and it will).
        ⁃       Possible “counter argument” — “but what about neuroscience? Don’t ask too much of yourself cause habits are hard to break, science says this and this about habits and exerting yourself and you don’t NEED to be PERFECT and you are good enough and healing is a SLOW PROCESS and blah blah blah new agey comfortable politically correct excuse after excuse that obviously limits potential when blindly believed”. The root of these arguments are a frame of neuroscience and biological processes being too powerful to harness. That you are literally too weak to effect uncomfortable change against neural circuitry. Says who? With what proof? Are psychology and neuroscience 100% figured out enough to give strong conclusive weight behind such arguments? Plus the view of “tyranny of neuroscience and biology” in effect amounts to “the world is a deterministic physical equation and process that traces itself independent of any free will, free will is just an illusion for a physical process acting through mere physical equations, so in the end you’re COMPLETELY POWERLESS against a physical equation of nature that is playing out deterministically.” What an awful view, but also unprovable. FAITH towards free will comes in here. And such faith against this naive nihilism is natural and intuitive anyway. Think about this: if the world is playing itself out regardless of you and your actions, why don’t you do nothing since it won’t matter anyway? Notice that YOU NEVER KNOW the state of the world anyway and ANY ACTION reflects an unprovable assumption (some more likely statistically/rationally/aesthetically than others though). Even assuming the world as a deterministic process, resigning to that, saying it doesn’t matter how you act, and letting that limit your effort is assuming something about the world, namely that you’re not good enough anyway and the world will screw you over anyway, that failure will physically happen anyway. But you don’t know that. YOU NEVER KNOW THE TRUE STATE OF THE WORLD. Rationally by far the best thing, given a strong value, is to assume with certainty you can and will bring it about, by putting ENOUGH effort. This is also rationally obvious, through history and individual experience.

Add more disciplinary restrictions than logically necessary
        ⁃       knowledge gap of unconscious processes
        ⁃       A better mindset and frame in itself

Christ as light. Overcomes “darkness”.
        ⁃       Light ALWAYS, PHYSICALLY overcomes darkness. Only way to see darkness is to turn gaze from light to darkness
        ⁃       Darkness misleads, you don’t know where you’re going, run into danger without knowing it, or seeing what is dangerous and what is not. Absence of colour. Renders one susceptible to infinite amounts of manipulations and false truths, due to an inherent inability to perceive the world as it is; an inability to perceive and understand what is True and Good. Light illuminates. Shows you the truth in the world. Colours the world with values. Light shows colour which reflects a TRUE property of the world, as in physics. Darkness is a tool used by deceivers and predators. Light is used by innovators and guides. Light/day shines when an area of Earth is turned towards the powerful, massive sun it revolves around. Darkness/night is when it turns away.
        ⁃       Christ is this “Light”. Believe in Christ -> see, absorb, and radiate light. (Perceive, believe/know, act)
        ⁃       Light as physical concept entwined with world (links to Christ as “becoming flesh”, understandable by human perception, logic, and action)
        ⁃       Father is intrinsically “unseeable”. But knowable. Christ is both knowable and seeable. Father logically MUST be unknowable (imagine an all-powerful God comprehensible by a human brain. A contradiction.) Christ links to a deep, immediate aesthetic perception and knowledge of the truth. Not rationally explainable all the way down. One does not need to understand everything about some concept to know it holds. Linking to Pascal’s thoughts, Christ allows for a correct, tuned “finesse” perception linked with intuitively understandable human experience rather than completely geometric manipulation all the way to the bottom.  But that mere fact allows for a much better tuning, lens, and direction of and for geometric/rational thinking. And since Christ is intimately, strongly, never deviating from the Father, we can “know” the Father, know the right thing and ways to act etc, without UNDERSTANDING every single detail (unachievable no matter what about literally ANYTHING). Christ then is the light showing us the Truth even if we don’t understand everything about the workings of light or physics or reasons why, but we perceive a truth regardless. Not just the truth, but also ACTIONS (link to “transcendental intuition” for correct action despite inability to rationally explain. Direct battle of adherence to Spirit vs. Pleasant comfortable grey “world” and it’s baseness - see Sodom and Gomorrah and the wife turning into stone)

God made Adam through “dust” and His breath
        ⁃       earthly/worldly (dust) and God’s breath (divine element)
        ⁃       Conceptual separation but also entwinement between physical world and realm of God
        ⁃       Distinct from animals
        ⁃       Breath -> force which moves dust in the direction of breath. Otherwise dust is merely swept aimlessly by the random wind of the physical world.
        ⁃       Life - deliberate movement, organization, sustainment of dust.  Spiritual death — return to formless dust, subject only to physics. Worldly death - literal transformation into dust. “Eternal life” — transcendence from mere transformation into dust and dynamics of being pushed around by everything else.

God influences world and history, for our easier access to the light.

Christ wipes feet of disciples with his own towel. Feet -> tool for movement. Disciples have feet washed; despite extensive effort, movement (walking), they are never filthy or worn down or rendered wary from their effort; the Lord always renders them clean, pure, free from dirt.

Christ as Lord — worldly concept of lord/powerful authority. One to obey and emulate. Whose directions we must follow. A man cannot serve two masters.
        ⁃       Christ as Lord washing feet of disciples -> disciples should emulate clear example (intuitive to humans due to human flesh of Christ) and wash one another’s feet. Help each other that they may not be weary or injured, and may continue walking in the direction of the Lord. Disciples deliberately “putting out” purifying water. (Christ as “living water” - source of aforementioned water - purifying of dirt, source of cleanliness and purity, sustainer of every living thing and their energy, always tapped upon and for drink so disciples will never be thirsty)

“happy are ye if you do (those things of the Lord)”. Never an inconvenience or “deep unpleasant sacrifice”. Happiness and immense, intrinsic motivation.

“He that receiveth whom ever I sent receiveth me” be good to people, especially those who have intuitively discernible light. No matter who they are, or their societally placed position or perception 

“Love one another; as I have loved you…” love transcending mere “exchange” or utility or convenience
        ⁃       such love as indication of discipleship recognizable to “ALL men”

Use your free will to take you out of contexts and directions that limit and weaken your free will, and use it to take you towards those that reinforce its power and usage.

The brain obeys and directs itself to what it really wants at any point in time. What you really want at any point in time is a choice. Conflicts between one’s competing desires are perceivable by the individual. At any given point they choose which one to follow, and thus preemptively determine and carve the path of the path of the future (to some temporal extent). At any point in time, the conflict is perceivable, and the choice can be made.

Aesthetic value having an objective part and a subjective part. Not purely subjective or determined by some “individual opinion”. Value is emanated by God who is consistent. Some inherent clear ordering of art. Bach > random rave music. Da Vinci > generative DallE art. Individuals are unique -> some subjectivity directed by your uniqueness.

Despite immense compute power of OpenAI-esque “huge brute force pretrained models on immense/near complete data repositories and knowledge bases”, there are limits to “stringent, specific objective function” training. Can be expert at a limited set of things. Perhaps even a few at the same time (limited multimodal learning). Restricted to effectiveness in restricted, more basic, internal digital world. Human effectiveness in inherently extremely multimodal and vague real world is very different. Simple specific objective functions cannot simply result in human emergencies such as culture, sustainable civilization, general condensation of different experiences and modes to generate creative ideas, know “where to look” in a general creative sense, and act coherently, quickly, across an extreme amount of “separate” specific domains, and juggle relevant information and skills for all of them, recognize general patterns everywhere in “unsupervised” fashion to link to everything else they know, maintain such an immense condensation of knowledge, ideas, beliefs, and a gateway to physical processes and homeostatic regulation. Immensely more complicated as the real world is infinitely more complicated than a digital ideal such as “language only environment” for chatGPT. Aforementioned emergences can only arise with complex, highly integrated systems coherently processing crazy amounts of separate modes/domains and interface with the real world.

Civilization, sharing of ideas (both in present, and across time and space — even generations apart) are important processes for humans to quickly gain basic knowledge and build upon it and achieve more/be more successful. Like multiple separate and distinct/unique processes but that are still correlated. Such that ideas and survival transcend time and space, even beyond individual death, through an iterative (possibly chaotic) process. Ability to occupy very many distinct positions in time and space while maintaining an alignment is a very powerful ability. Humanity as “unified entity”.
        ⁃       strangely, conflict between groups of humans is a very strong driver for creative innovation and refinement (motivation for intellectual breakthroughs in many forms to trounce powerful competition). Like a process that accelerates itself, makes itself powerful.

Concept of language immensely powerful as intrinsic unified consolidation of human knowledge, ideas, expression, logic, desires. “Language function” learned brute force by LLMs can thus intrinsically hit many powerful intellectual birds with one stone. But not all.

possibility of theoretical highly mathematically and logically inclined language, able to represent higher level concepts (physics etc., theories). Conversation and linking of ideas by communicating in the language (with randomness/exploration). Could learning a “language model” for such a language be another leap for AI? How could we construct such a language?


Application of intelligence for self interest and self glory, leads to openAI etc., institutions reaping the benefits and exploiting the shared discoveries of others, who did the lion’s share of intellectual heavy lifting, for their own ever cascading and monopolizing power, to the detriment of everyone else (the vast majority of people). Good ideas shared for the sake of glory, perceptions of “intellectual superiority”, and respect; picked up by businessmen orders of magnitude less “geometrically” creative or intelligent, but far more ambitious, Machiavellian, extroverted, and astute at communication. Folks who would destroy everything and everyone for the sake of misled conceptions of power, the idea of which becomes progressively more twisted, needlessly brutal, immensely useless, and addictive the more of it one obtains. Like an endless mountain whose weight crushes its foundation more and more as mighty, powerful stones are added, causing the surface on which it stands to cave faster than the mountain grows, leading to an infinite mountain sitting in an even more infinitely deep pit. These misled leaders and authority figures would have no power for such destruction if these powerful ideas were not widely disseminated for the sake of the inventor’s glory. Is knowledge “openness” like this really a virtue? In the end those who use it for power, such as openAI, end up “closing” their knowledge to the world, and though their high position affords these institutions much more potential and motivation to innovate than anyone else, they keep their discoveries to themselves to avoid the mistakes of those they copied: to prevent competition. They let people use their inventions, but never expose their blueprints; people end up depending on these powerful vampires. But clearly some knowledge and discoveries should, sometimes must, be disseminated. The distinction is this: some things we need, and other things we want or are merely “nice to have” for the sake of pleasure, convenience, distraction, numbing of insecurities, or circlejerks of glory and selfishness. AI is “nice to have”. I heard somewhere (I think Yuvi telling me about a theory from the book “Sapiens”) that the move of human groups from hunter gatherer  tribes to agriculturally organized led to significant declines in quality of life and, ironically, relative inefficiency (different than “absolute production”) compared to the hunter gatherer mode of living. That is the same concept, a “nice to have” ambition, leading to a “less satisfied” state which ended up spilling out and carving the history of humanity for the following millennia. Today, the ubiquitous orientation of efforts to sell for “nice to have”, and the endless, accelerating demand for such products and distractions, can and probably will amount to chaos we will be too distracted to foresee or even perceive when it does manifest.

A person’s passions, pride, and their sense of importance, are by definition the most meaningful thing to communicate with them. For any individual, there are few interactions more validating, satisfying, evidential of one’s power and autonomy to effect valuable change in the world, and overall immensely meaningful than the perception of someone else perceiving their passion and being touched and filled with it; a perception of genuine interpersonal alignment on the fronts most significant to oneself; a true look of awe, wonder, appreciation, and understanding on the face of the person to whom the individual’s work or values are being communicated. It is almost a direct conduit to a light, a positive mode seldom seen in the lives of many. It is no surprise that such rare yet meaningful interactions lead to acts of goodness by the end of them, as tokens of sincere gratitude. Of course such interactions generally are possible when the listener is open, honest, genuine, and good spirited, without devious intention (exceptionally, some scheming psychopaths with acting talent have this in their repertoire of “strategies”; exercise extreme caution with such folk, and this can be partially countered by being selfless yourself and interested in the listener as well). Things that are meaningful to someone follow a general component common among all humanity, but with a significant unique touch tied to the individual. Seek out light in eyes and voice, and follow the traces of light; selflessness is the key here. Exploring semi randomly can lead somewhere the other person finds fascinating; one can never know or at least be sure off the bat of the exact nature of the things truly meaningful to someone. In the end, conversation under passion is important to them and me. To me, the listener, it is the only meaningful conversation. Conversation only for the sake of mere pleasure or utility is unfulfilling and uninteresting. Note, however, that such interactions connote a “power differential” as framed by the speaker. Be aware of this. In the end this factors mostly for Machiavellian purposes. For selfless intention, it is insignificant. And I want to direct my actions selflessly, not manipulatively.

“shallow” presentation as foot in the door for greater and more valuable interaction.